\documentclass{beamer}

\usepackage{beamerthemesplit}
\usepackage{graphicx}
\usepackage{color, natbib, hyperref}
\usepackage{bibentry}
\nobibliography*

% define colors
\definecolor{jblue}  {RGB}{20,50,100}
\definecolor{ngreen} {RGB}{98,158,31}

%theme

\usetheme{boxes} 
%\usecolortheme{seahorse} 
\setbeamertemplate{items}[default] 
%\setbeamercovered{transparent}
\setbeamertemplate{blocks}[rounded]
\setbeamertemplate{navigation symbols}{} 
% set the basic colors
\setbeamercolor{palette primary}   {fg=black,bg=white}
\setbeamercolor{palette secondary} {fg=black,bg=white}
\setbeamercolor{palette tertiary}  {bg=jblue,fg=white}
\setbeamercolor{palette quaternary}{fg=black,bg=white}
\setbeamercolor{structure}{fg=jblue}
\setbeamercolor{titlelike}         {bg=jblue,fg=white}
\setbeamercolor{frametitle}        {bg=jblue!10,fg=jblue}
\setbeamercolor{cboxb}{fg=black,bg=jblue}
\setbeamercolor{cboxr}{fg=black,bg=red}

% reduce space before/after equations
\expandafter\def\expandafter\normalsize\expandafter{%
    \normalsize
    \setlength\abovedisplayskip{1pt}
    \setlength\belowdisplayskip{1pt}
    \setlength\abovedisplayshortskip{1pt}
    \setlength\belowdisplayshortskip{1pt}
}

% set colors for itemize/enumerate
\setbeamercolor{item}{fg=ngreen}
\setbeamercolor{item projected}{fg=white,bg=ngreen}

% set colors for blocks
\setbeamercolor{block title}{fg=ngreen,bg=white}
\setbeamercolor{block body}{fg=black,bg=jblue!10}

% set colors for alerted blocks (blocks with frame)
\setbeamercolor{block alerted title}{fg=white,bg=jblue}
\setbeamercolor{block alerted body}{fg=black,bg=jblue!10}
\setbeamercolor{block alerted title}{fg=white,bg=dblue!70} % Colors of the highlighted block titles
\setbeamercolor{block alerted body}{fg=black,bg=dblue!10} % Colors of the body of highlighted blocks

% set the fonts
\usefonttheme{professionalfonts}

\setbeamerfont{section in head/foot}{series=\bfseries}
\setbeamerfont{block title}{series=\bfseries}
\setbeamerfont{block alerted title}{series=\bfseries}
\setbeamerfont{frametitle}{series=\bfseries}
\setbeamerfont{frametitle}{size=\Large}
\setbeamerfont{block body}{series=\mdseries}
\setbeamerfont{caption}{series=\mdseries}
\setbeamerfont{headline}{series=\mdseries}


% set some beamer theme options
\setbeamertemplate{title page}[default][colsep=-4bp,rounded=true]
\setbeamertemplate{sections/subsections in toc}[square]
\setbeamertemplate{items}[circle]
\setbeamertemplate{blocks}[width=0.0]
\beamertemplatenavigationsymbolsempty

% Making a DAG
\usepackage{tkz-graph}  
\usetikzlibrary{shapes.geometric}
\usetikzlibrary{positioning}
\tikzstyle{VertexStyle} = [shape            = rectangle,
                               minimum width    = 6ex,%
                               draw]
 \tikzstyle{EdgeStyle}   = [->,>=stealth']      

% Define block styles
\tikzstyle{f} = [rectangle, draw, fill=blue!20, 
    text width=3em, text badly centered, node distance=1.75cm]
\tikzstyle{message} = [rectangle, draw, fill=green!20, 
    text width=3em, text centered]
\tikzstyle{io} = [draw, circle,fill=red!20, node distance=2cm,
    minimum height=2em]
\tikzstyle{line} = [draw, -latex']

% Math macros
\newcommand{\cD}{{\mathcal D}}
\newcommand{\cF}{{\mathcal F}}
\newcommand{\todo}[1]{{\color{red}{TO DO: \sc #1}}}

\newcommand{\reals}{\mathbb{R}}
\newcommand{\integers}{\mathbb{Z}}
\newcommand{\naturals}{\mathbb{N}}
\newcommand{\rationals}{\mathbb{Q}}

\newcommand{\ind}[1]{1_{#1}} % Indicator function
\newcommand{\pr}{\mathbb{P}} % Generic probability
\newcommand{\ex}{\mathbb{E}} % Generic expectation
\newcommand{\var}{\textrm{Var}}
\newcommand{\cov}{\textrm{Cov}}

\newcommand{\normal}{N} % for normal distribution (can probably skip this)
\newcommand{\eps}{\varepsilon}
\newcommand\independent{\protect\mathpalette{\protect\independenT}{\perp}}
\def\independenT#1#2{\mathrel{\rlap{$#1#2$}\mkern2mu{#1#2}}}

\newcommand{\convd}{\stackrel{d}{\longrightarrow}} % convergence in distribution/law/measure
\newcommand{\convp}{\stackrel{P}{\longrightarrow}} % convergence in probability
\newcommand{\convas}{\stackrel{\textrm{a.s.}}{\longrightarrow}} % convergence almost surely

\newcommand{\eqd}{\stackrel{d}{=}} % equal in distribution/law/measure
\newcommand{\argmax}{\arg\!\max}
\newcommand{\argmin}{\arg\!\min}

\newcommand{\bit}{\begin{itemize}}
\newcommand{\eit}{\end{itemize}}


\mode<presentation>

\title[Simple Random Sampling: Not So Simple]{Simple Random Sampling: Not So Simple}
\author{Kellie Ottoboni \\ with Philip B.~Stark and Ron Rivest}
\institute[]{Department of Statistics, UC Berkeley\\Berkeley Institute for Data Science}
\date{Qualifying Exam \\ January 23, 2017}

\begin{document}

\frame{
\titlepage
\vfill
\begin{columns}[T]
\begin{column}{.5\textwidth}
\begin{center}
\vspace{25pt}
\includegraphics[width=\textwidth]{fig/logo/dept1.pdf}
\end{center}
\end{column}
\begin{column}{.3\textwidth}
\begin{center}
\end{center}
\end{column}
\begin{column}{.3\textwidth}
\begin{center}
\includegraphics[width=0.9\textwidth]{fig/logo/BIDS.png}
\end{center}
\end{column}
\end{columns}
}



\section[Introduction]{Introduction}

\frame{
\frametitle{Cute baby}

}

\frame{
\frametitle{Scary werewolf}
}


%\frame
%{
%  \frametitle{PRNGs}
%  
%  \textbf{Pseudorandom number generator:} a deterministic algorithm that produces sequences that are computationally indistinguishable from the uniform distribution
%%\begin{center}
%%\begin{itemize}
%%\item \textbf{Simple random sampling:} drawing $k \le n$ items from a population of $n$ items, in such a way that each of the $n \choose k$ subsets of size $k$ is equally likely.
%%\item Difficult to obtain truly random samples. Instead, use \textbf{pseudorandom number generators (PRNGs)} to select items
%%\item \textbf{Pseudorandom}: computationally indistinguishable from the uniform distribution
%%\end{itemize}
%%\end{center}
%%
%%Good PRNGs produce pseudorandom sequences. Do they give simple random samples with equal probabilities?
%
%}

\frame{
\frametitle{Pigeons and Pigeonholes}

\begin{theorem}[Pigeonhole Principle]
If there are $n$ pigeonholes and $m>n$ pigeons, then there exists at least one pigeonhole containing more than one pigeon.
\end{theorem}
\begin{figure}[htbp]
\begin{center}
\includegraphics[width = .3\textwidth]{fig/TooManyPigeons.jpg}
\end{center}
\tiny \href{https://commons.wikimedia.org/w/index.php?curid=4658682}{(Wikipedia)}
\end{figure}

\pause
\begin{corollary}[Too few pigeons]
If ${n \choose k}$ is greater than the size of a PRNG's state space, then the PRNG cannot possibly generate all samples of size $k$ from a population of $n$.
\end{corollary}
}


\frame{
\frametitle{Pigeons and Pigeonholes}

Does it matter in practice? 
\pause

\vspace{20pt}

Period of 32-bit linear congruential generators (e.g. RANDU): $2^{32} \approx 4 \times 10^9$ \\
Samples of size $10$ from $50$: ${50 \choose 10} \approx 10^{10}$ \\
\textbf{More than half of samples cannot be generated}
\vspace{20pt}
\pause

Period of Mersenne Twister (standard PRNG in Statistics): $2^{32 \times 624} \approx 2 \times 10^{6010}$ \\
Permutations of $2084$ objects: $2084! \approx 3 \times 10^{6013}$\\
\textbf{Less than $0.01\%$ of permutations can be generated}


}


\frame{
\frametitle{Impossibility bounds}
}

\frame{
\frametitle{Simulation showing nonuniformity of MT}
}

\section[Sampling algorithms]{Sampling Algorithms}

\frame{
\frametitle{PIKK}
\bit
\item Show algorithm
\item which software uses PIKK?
\eit
}

\frame{
\frametitle{Better algorithms}

}

\frame{
\frametitle{Sampling simulations with MT}

}

\section[PRNGs]{Pseudorandom Number Generators}

\frame{
\frametitle{What makes a PRNG}

\bit
\item Fast and memory efficient
\item Mimics a random sequence (statistically indistinguishable)
\item Unpredictable. this is different from random - if it's deterministic, then it's predictable to some degree
\item Jump-ahead feature so we can efficiently skip random numbers, generate multiple streams for paraellel applications
\eit
}



\frame{
\frametitle{Linear Congruential Generators}
Explain
}

\frame{
\frametitle{The good, the bad, and the ugly}
\begin{block}{(Knuth, 1997)}
``Random numbers should not be generated with a method chosen at random.''
\end{block}
\pause


\begin{figure}[htbp]
\begin{center}
\includegraphics[width = 0.4\textwidth]{fig/randu.png}
\end{center}
     Triples of RANDU lie on 15 planes in 3D space \\ 
     $x_{n+1} = (65539 x_{n}) \mod 2^{31}$ \\
\footnotesize{(Wikipedia)}
\end{figure}
}

\frame{
\frametitle{Linear Congruential Generators}
\bit
\item Fast to compute
\item Some are more random than others - show constants, RANDU
\item Not unpredictable. We only need 2 values to determine the constants
\item Possible to do jump ahead
\eit
}



\frame{
\frametitle{Mersenne Twister}
explain. state of the art in statistics
}

\frame{
\frametitle{Mersenne Twister}
\bit
\item Fast to compute but has a large state space, not the most memory efficient
\item Random to a good degree
\item Completely predictable after we've seen 623 values
\item No good jump ahead feature
\eit
}



\frame{
\frametitle{A better alternative}

\textbf{One solution:} Find a class of PRNGs with infinite state space
}

\frame{
\frametitle{Hash function PRNGs}

Describe how hash functions work

\begin{center}
\resizebox{10cm}{!}{    
\begin{tikzpicture}[node distance = 1cm, auto, scale = 0.5]
    % Place nodes
    \node [io] (IV) {IV};
    \node [f, right of=IV] (f1) {f};
    \node [f, right of=f1] (f2) {f};
%    \node [f, right of=f2, minimum width = 0cm, height = 0cm] (invisible) {};
    \node [f, right of=f2, node distance=3cm] (fn1) {f};
    \node [f, right of=fn1] (fend) {f};
%    \node [f, right of=invisible, node distance=3cm] (fend) {f};
    \node [f, right of=fend] (g) {g};
    \node [io, right of=g] (hx) {$h(x)$};
    \node [message, above of=f1] (m1) {$x_1$};
    \node [message, above of=f2] (m2) {$x_2$};
    \node [message, above of=fn1] (mn1) {$x_{n-1}$};
    \node [message, above of=fend] (mend) {$x_n$};
    \node [message, above right = of m1, above left = of mend, minimum width = 5cm] (message) {$x$};
    % Draw edges
    \path [line] (IV) -- (f1);
    \path [line] (f1) -- (f2);
    \path [line] (message) -- (m1);
    \path [line] (message) -- (m2);
    \path [line] (message) -- (mn1);
    \path [line] (message) -- (mend);
    \draw [-,dotted] (m2) -- (mn1);
    \path [line] (m1) -- (f1);
    \path [line] (m2) -- (f2);
     \path [line] (mn1) -- (fn1);
    \path [line] (mend) -- (fend);
    \path [line] (fend) -- (g);
    \path [line,dashed] (f2) -- (fn1);
    \path [line] (fn1) -- (fend);
    \path [line] (g) -- (hx);
\end{tikzpicture}
}
\end{center}


Cryptographic hash functions:
\begin{itemize}
\item computationally infeasible to invert
\item difficult to find two inputs that map to the same output
\item small input changes produce large, unpredictable changes to output
\item resulting bits are uniformly distributed
\end{itemize}

}


\frame{
\frametitle{Hash function PRNGs}
Describe procedure with counter
\bit
\item Fast: depends on well-tested hash function code. Number of hashes may slow it down, with the benefit of adding more unpredictability.
\item Memory efficient: Store a seed and a counter -- just like 25 digits
\item Unpredictable: small changes to input produce large unpredictable changes to output. The only way to figure out the sequence is to know the hashing function
\item Jump ahead: just add to the counter.
\eit
}

\section[Results]{Results}

\frame{
\frametitle{Simulations}
\bit
\item Compare PIKK/good sampling algo, MT/CSPRNG/LCG/KISS, different seeds
\item For the unique sample frequencies, report
\bit
\item Range statistic p-value
\item Chi-squared test p-value
\eit
\item Do this for various numbers of samples - 1e6 up to 1e8. This relates to equidistribution
\eit
}


\frame{
\frametitle{Choice of seed}

Preliminary results: the distribution of simple random samples is less uniform if you use a stupid seed

\begin{table}[htdp]
\begin{center}
\begin{tabular}{r|c|c|}
 & $p$-value& $p$-value\\
PRNG &  (seed = 100) &  (seed = 233424280) \\
\hline
RANDU & 0  & 0 \\
Super-Duper LCG & 0.1798 & 1 \\
Mersenne Twister*  & 0.0858 & 0.4741 \\
Mersenne Twister & 0.1996 & 0.6143 \\
SHA-256 PRNG & 0.1710 & 0.8584
\end{tabular}
\end{center}
\label{default}
\end{table}%

\small{* using \texttt{np.random.choice} to sample}

}



\frame{
\frametitle{Open questions}
\begin{itemize}
\itemsep10pt
\item Do ``good'' PRNGs produce all samples with equal probability? All permutations?
\pause
\item Do departures from uniformity introduce bias?
\pause
\item Replace the default PRNGs in Python
\url{https://www.github.com/statlab/cryptorandom}
\pause
\item Results apply more broadly to computer simulations: permutation tests, bootstrapping, MCMC, etc.
\end{itemize}

}


\end{document}
